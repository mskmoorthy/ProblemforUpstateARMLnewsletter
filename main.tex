\documentclass{article}
\usepackage[utf8]{inputenc}

\title{ProblemforUpstateARMLnewsletter}
\author{mukkai.krishnamoorthy }
\date{November 2019}

\begin{document}

\maketitle

\section{Problem}
(Adapted from IIT Entrance Examination)
Solve for x (as a closed form expression using logarithms, radicals and well algebraic numbers).
\begin{equation}
    {12}^x+9^x~=~{16}^x 
\end{equation}
\begin{newpage}
\section{Solution}
$
    {12}^x+9^x~=~{16}^x $
    \\
    Dividing the above equation by $9^x$, we get \\
    ${(\frac{12}{9})}^x+1={(\frac{16}{9})}^x$, which simplifies to\\
    ${(\frac{4}{3})}^x+1={(\frac{4}{3})}^{2x}$ \\
    Substituting $p$ for ${(\frac{4}{3})}^x$  in the above equation, we get\\
    $p+1=p^2$\\
    The solution of the above equation is $\Phi$ and $1-\Phi$ where \\
    $\Phi$ is the golden ratio = $\frac{1+\sqrt{5}}{2}$.\\
    $p={(\frac{4}{3})}^x= \Phi$ and $1 -\Phi$\\
    Taking logarithms, we get\\
    $x =\frac{\log(\Phi)}{\log(\frac{4}{3})} $\\
    Since $1-\Phi$ is negative, that solution will not appear (because log of negative numbers do not exist)

\end{newpage}
\end{document}
